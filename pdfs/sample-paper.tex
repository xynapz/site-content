\documentclass[12pt,a4paper]{article}

% Packages
\usepackage[utf8]{inputenc}
\usepackage[T1]{fontenc}
\usepackage{amsmath,amssymb,amsthm}
\usepackage{graphicx}
\usepackage{geometry}
\usepackage{listings}
\usepackage{xcolor}

\usepackage{hyperref}

\geometry{margin=1in}

\newtheorem{theorem}{Theorem}

\lstset{
    basicstyle=\ttfamily\small,
    keywordstyle=\color{blue},
    commentstyle=\color{gray},
    stringstyle=\color{red},
    numbers=left,
    numberstyle=\tiny\color{gray},
    frame=single,
    breaklines=true
}

\title{Sample Research Paper: Data Structures and Algorithms}
\author{Angel Dhakal}
\date{\today}

\begin{document}

\maketitle

\begin{abstract}
This is a sample LaTeX document demonstrating various features commonly used in academic papers. It includes mathematical equations, code listings, tables, and figures. This document can serve as a template for your academic writing.
\end{abstract}

\section{Introduction}

This document demonstrates the basic structure of an academic paper written in LaTeX. LaTeX is particularly well-suited for documents containing mathematical notation and technical content.

Some key advantages of LaTeX include:
\begin{itemize}
    \item Professional typesetting of mathematical formulas
    \item Automatic numbering and cross-referencing
    \item Consistent formatting throughout the document
    \item Excellent bibliography management
\end{itemize}

\section{Mathematical Notation}

\subsection{Basic Equations}

The quadratic formula is given by:
\begin{equation}
    x = \frac{-b \pm \sqrt{b^2 - 4ac}}{2a}
    \label{eq:quadratic}
\end{equation}

Euler's identity, considered one of the most beautiful equations in mathematics:
\begin{equation}
    e^{i\pi} + 1 = 0
\end{equation}

\subsection{Advanced Mathematics}

The Taylor series expansion of a function $f(x)$ around point $a$ is:
\begin{equation}
    f(x) = \sum_{n=0}^{\infty} \frac{f^{(n)}(a)}{n!}(x-a)^n
\end{equation}

Matrix operations in linear algebra:
\begin{equation}
    A = \begin{pmatrix}
        a_{11} & a_{12} & a_{13} \\
        a_{21} & a_{22} & a_{23} \\
        a_{31} & a_{32} & a_{33}
    \end{pmatrix}
\end{equation}

\section{Code Examples}

Here's a simple Python implementation of binary search:

\begin{lstlisting}[language=Python, caption=Binary Search Algorithm]
def binary_search(arr, target):
    left, right = 0, len(arr) - 1

    while left <= right:
        mid = (left + right) // 2

        if arr[mid] == target:
            return mid
        elif arr[mid] < target:
            left = mid + 1
        else:
            right = mid - 1

    return -1
\end{lstlisting}

\section{Tables and Data}

Table~\ref{tab:complexity} shows the time complexity of common algorithms.

\begin{table}[h]
\centering
\begin{tabular}{|l|l|l|l|}
\hline
\textbf{Algorithm} & \textbf{Best} & \textbf{Average} & \textbf{Worst} \\
\hline
Bubble Sort & $O(n)$ & $O(n^2)$ & $O(n^2)$ \\
Quick Sort & $O(n\log n)$ & $O(n\log n)$ & $O(n^2)$ \\
Merge Sort & $O(n\log n)$ & $O(n\log n)$ & $O(n\log n)$ \\
Binary Search & $O(1)$ & $O(\log n)$ & $O(\log n)$ \\
\hline
\end{tabular}
\caption{Time Complexity of Common Algorithms}
\label{tab:complexity}
\end{table}

\section{Theorems and Proofs}

\begin{theorem}[Pythagorean Theorem]
In a right triangle, the square of the hypotenuse is equal to the sum of squares of the other two sides:
\[
c^2 = a^2 + b^2
\]
\end{theorem}

\begin{proof}
This can be proven using geometric construction or algebraic manipulation. The proof is left as an exercise to the reader.
\end{proof}

\section{Cross-References}

We can reference Equation~\ref{eq:quadratic} from earlier in the document. Similarly, we can refer to Table~\ref{tab:complexity} or any numbered element.

\section{Lists and Enumeration}

\subsection{Ordered List}
\begin{enumerate}
    \item First step: Initialize variables
    \item Second step: Process input
    \item Third step: Return result
\end{enumerate}

\subsection{Unordered List}
\begin{itemize}
    \item Data structures
    \item Algorithms
    \item Complexity analysis
    \item Design patterns
\end{itemize}

\section{Conclusion}

This sample document demonstrates the basic features of LaTeX for academic writing. LaTeX provides excellent support for mathematical typesetting, code listings, tables, and cross-referencing, making it ideal for technical and scientific documentation.

\subsection{Further Reading}

For more information on LaTeX, consult:
\begin{itemize}
    \item The LaTeX Project: \url{https://www.latex-project.org/}
    \item Overleaf Documentation: \url{https://www.overleaf.com/learn}
    \item Stack Exchange TeX: \url{https://tex.stackexchange.com/}
\end{itemize}

\end{document}
